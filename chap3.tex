%!TEX encoding = UTF-8 Unicode
\documentclass[11pt]{report}
\usepackage[margin = 1in]{geometry}

\usepackage{amsmath}
\usepackage{amssymb}
\usepackage{mleftright}
\usepackage{enumitem}
\usepackage{textcomp, gensymb}
\usepackage{tikz, tikz-cd}
\usepackage{bbding}
\usepackage{tabularx}
\usepackage{nicematrix}
\usepackage{extarrows}
\usepackage{lastpage}
\usepackage{hyperref}

\usepackage{fancyhdr}  % Header and Footer formatting

\usetikzlibrary{decorations.markings}

\usepackage{multicol}

\hypersetup{colorlinks = true,
            linkcolor = blue,
            citecolor = red,
            urlcolor = teal}

\pagestyle{fancy}
\renewcommand{\headrulewidth}{0.4pt}
\renewcommand{\footrulewidth}{0.4pt}
\setlength{\headheight}{18pt}

% Header and Footer Information
\lhead{\small\emph{Baitian Li}}
\chead{}
\rhead{\textsc{Complex Analysis}}
\lfoot{\today}
\cfoot{}
\rfoot{\thepage\ of \pageref{LastPage}}  % Counts the pages.

\makeatletter % This provides a total page count as \ref{NumPages}
\AtEndDocument{\immediate\write\@auxout{\string\newlabel{NumPages}{{\thepage}}}}
\makeatother

%\lineskiplimit=-\maxdimen\relax

\usepackage{amsthm}  % This will create the Problem environment
\newtheorem*{lemma}{Lemma}
\newtheorem*{theorem}{Theorem}
\newtheoremstyle{mythm}%
    {12pt}{}%
    {\sffamily}{}%
    {\bfseries \sffamily}{.}%
    {.5em}%
    {\thmname{#1}\thmnumber{ #2}\thmnote{ #3}}

\theoremstyle{mythm}

\expandafter\let\expandafter\oldproof\csname\string\proof\endcsname
\let\oldendproof\endproof
\renewenvironment{proof}[1][\proofname]{%
  \oldproof[\normalfont \bfseries #1]%
}{\oldendproof}

\renewcommand*{\proofname}{Proof}

\newtheoremstyle{myans}%
    {12pt}{}%
    {}{}%
    {\bfseries}{.}%
    {.5em}%
    {\thmname{#1}\thmnumber{ #2}\thmnote{ #3}}

\theoremstyle{myans}
\newtheorem*{answer}{Answer}

\setlist[enumerate]{noitemsep, topsep = 0.2em}
\setlist[enumerate]{label = {\bfseries \arabic*.}}
\setlist[enumerate, 2]{label = {(\alph*)}}
\setlist[description]{topsep = 0.2em, listparindent = \parindent, font = \normalfont,
  itemsep = 0em}

\newcommand{\mi}{\mathrm{i}}
\newcommand{\me}{\mathrm{e}}
\newcommand{\dd}{\mathop{}\!\mathrm{d}}

\newcommand{\bbC}{\mathbb{C}}
\newcommand{\bbD}{\mathbb{D}}
\newcommand{\bbR}{\mathbb{R}}

\newcommand{\norm}[1]{\|#1\|}

\renewcommand{\Re}{\operatorname{Re}}
\renewcommand{\Im}{\operatorname{Im}}

\DeclareMathOperator{\res}{res}

\allowdisplaybreaks

\begin{document}

\setcounter{chapter}{2}
\chapter{Meromorphic Functions and the Logarithm}

\section{Exercises}

\begin{enumerate}
  \item Using Euler's formula
  \[ \sin \pi z = \frac{\me^{\mi \pi z} - \me^{-\mi \pi z}}{2\mi}, \]
  show that the complex zeros of $\sin\pi z$ are exactly at the integers, and that they are each of order $1$.
  Calculate the residue of $1/ \sin \pi z$ at $z = n \in \mathbb Z$.
  \begin{proof}
    For the zeroes $\sin \pi z = 0$, we have $\me^{2\mi \pi z} = 1$, the solutions are exactly
    $z=n$. And near the integer $n$ we have
    \begin{align*}
      \sin \pi z &= (\me^{2\mi \pi z} - 1)\frac{\me^{-\mi \pi z}}{2\mi}\\
      &= (\me^{2\mi \pi (z-n)} - 1)\frac{\me^{-\mi \pi z}}{2\mi}\\
      &= 2\mi \pi (z-n)\frac{\me^{-\mi \pi z}}{2\mi}(1 + O(z-n))\\
      &= (z-n)\left((-1)^n\pi  + O(z-n)\right).
    \end{align*}
    So we have $\res_n (1/\sin \pi z) = (-1)^n / \pi$.
  \end{proof}
  \item Evaluate the integral
  \[ \int_{-\infty}^{\infty} \frac{\dd x}{1+x^4}. \]
  \begin{answer}
    Let $\omega = \me^{\mi \pi/4}$, we have
    \[ \frac 1{z^4+1} =\frac 1{(z-\omega)(z-\omega^3)(z-\omega^5)(z-\omega^7)}. \]
    Therefore, the function has four poles with order $1$.
    Consider the semicircle $\gamma_R$ with radius $R$ large enough.
    The residue formula gives
    \[ \int_{\gamma_R} \frac{\dd z}{z^4+1} = 2\pi \mi \left(\res_{\omega} \frac 1{z^4+1}
    + \res_{\omega^3} \frac 1{z^4+1} \right). \]
    We have
    \begin{align*}
      \res_\omega \frac 1{z^4+1} &= \lim_{z\to \omega} \frac{z-\omega}{z^4+1}\\
      &= \frac1{(\omega-\omega^3)(\omega-\omega^5)(\omega-\omega^7)}\\
      &= \frac1{4\omega^3}.
    \end{align*}
    Similarly, we have
    \[ \res_{\omega^3} \frac1{z^4+1} = \frac 1{4\omega}. \]
    Therefore,
    \begin{align*}
      &\quad 2\pi \mi \left(\res_{\omega} \frac 1{z^4+1}
      + \res_{\omega^3} \frac 1{z^4+1} \right)\\
      &= \frac {\pi}{2} \mi (\omega^{-1} + \omega^{-3})\\
      &= \frac{\sqrt 2}{2} \pi.
    \end{align*}
    For the outer semicircle $C_R^+$, we have
    \[ \left| \int_{C_R^+} \frac{\dd z}{z^4+1} \right| = O\left( R \cdot \frac 1{R^4} \right) \to 0. \]
    So we have
    \[ \int_{-\infty}^{\infty} \frac{\dd x}{x^4+1} = \frac{\sqrt 2}{2} \pi. \]
  \end{answer}
  \item Show that
  \begin{center}
    $\displaystyle \int_{-\infty}^\infty \frac{\cos x}{x^2+a^2} \dd x = \pi \frac{\me^{-a}}{a}$,
    \quad for $a>0$.
  \end{center}
  \begin{proof}
    Only need to compute the real part of the integration
    \[ \int_{-\infty}^\infty \frac{\me^{\mi x}}{x^2+a^2} \dd x. \]
    Consider the semicircle $\gamma_R$ with radius large enough. Since
    \[ \frac{\me^{\mi z}}{z^2+a^2} = \frac{\me^{\mi z}}{(z+\mi a)(z-\mi a)} \]
    has a pole at $\mi a$, residue formula gives
    \begin{align*}
      \int_{\gamma_R} \frac{\me^{\mi z}}{z^2+a^2}\dd z &= 2\pi \mi \res_{\mi a} \frac{\me^{\mi z}}{z^2+a^2}\\
      &= 2\pi \mi \lim_{z\to \mi a} \frac{\me^{\mi z}(z - \mi a)}{z^2+a^2}\\
      &= \frac{\pi \me^{-a}}{a}.
    \end{align*}
    For the outer semicircle $C_R^+$, we have
    \begin{align*}
      &\quad \left| \int_{C_R^+} \frac{\me^{\mi z}}{z^2+a^2}\dd z \right|\\
      & \leq \int_{C_R^+} \frac{|\me^{\mi z}|}{R^2 - a^2} \left|\dd z\right|\\
      & = \int_{C_R^+} \frac{\me^{-\Im z}}{R^2 - a^2} \left|\dd z\right|\\
      & = \int_0^\pi \frac{R\me^{-R\sin \theta}}{R^2 - a^2} \dd \theta\\
      &\leq \frac{2R}{R^2-a^2} \int_0^{\pi/2} \me^{-2R\theta/\pi} \dd \theta\\
      &\leq \frac{\pi}{R^2-a^2} \to 0.
    \end{align*}
    So we have
    \[ \int_{-\infty}^\infty \frac{\cos x}{x^2+a^2}\dd x = \pi \frac{\me^{-a}}{a}. \qedhere \]
  \end{proof}
  \setcounter{enumi}{5}
  \item Show that
  \[ \int_{-\infty}^\infty \frac{\dd x}{(1+x^2)^{n+1}} = \frac{1\cdot 3\cdots (2n-1)}
  {2\cdot 4\cdot 6\cdots (2n)} \cdot \pi. \]
  \begin{proof}
    Since
    \[ \frac 1{(z^2+1)^{n+1}} = \frac 1{(z-\mi)^{n+1}(z+\mi)^{n+1}} \]
    has pole of order $n+1$ at $\mi$, the semicircle $\gamma_R$ with $R$ large enough has
    has
    \begin{align*}
      \int_{\gamma_R} \frac{\dd z}{(z^2+1)^{n+1}} &= 2\pi \mi \res_{\mi} \frac 1{(z^2+1)^{n+1}}\\
      &= 2\pi \mi \lim_{z\to \mi} \frac 1{n!} \frac{\dd^n}{\dd z^n} \frac1{(z + \mi)^{n+1}}\\
      &= 2\pi \mi \lim_{z\to \mi} \frac 1{n!} \frac{(-n-1)(-n-2)\cdots (-2n)}{(z + \mi)^{2n+1}}\\
      &= 2\pi \mi \frac 1{n!} \frac{(-n-1)(-n-2)\cdots (-2n)}{(2\mi)^{2n+1}}\\
      &= \frac{(-n-1)(-n-2)\cdots (-2n)}{(-4)^{n} n!} \cdot \pi\\
      &= \frac{(2n)!}{n!^2 4^n} \cdot \pi\\
      &= \frac{1\cdot 3\cdots (2n-1)}
      {2\cdot 4\cdot 6\cdots (2n)} \cdot \pi. \qedhere
    \end{align*}
  \end{proof}
  \setcounter{enumi}{7}
  \item Prove that
  \[ \int_0^{2\pi} \frac{\dd \theta}{a + b\cos \theta} = \frac{2\pi}{\sqrt{a^2-b^2}} \]
  if $a>|b|$ and $a,b\in\bbR$.
  \begin{proof}
    We first analyze the poles of $1/(a + b\cos z)$. Since
    \[ \cos z = \frac{\me^{\mi z} + \me^{-\mi z}}2, \]
    the poles are at $a + b\cos z = 0$, i.e.,
    \[ \me^{\mi z} = \frac{-a \pm \sqrt{a^2-b^2}}{b}. \]
    Thus
    \begin{align*}
      \mi z &= 2\pi n \mi + \ln\left( \frac{-a \pm \sqrt{a^2-b^2}}{b} \right)\\
      &= (2n + 1)\pi \mi \pm \ln\left( \frac{a-\sqrt{a^2-b^2}}{b} \right),\\
      z &= (2n+1)\pi \pm \mi \ln\left( \frac{a-\sqrt{a^2-b^2}}{b} \right).
    \end{align*}
    Let
    \[ z_0 = \pi - \mi \ln\left( \frac{a-\sqrt{a^2-b^2}}{b} \right), \]
    at $z_0$ we have
    \begin{align*}
      a + b\cos z &= a + b\cos((z - z_0) + z_0)\\
      &= a + b(\cos z_0 - \sin z_0 \cdot (z-z_0) + O((z-z_0)^2))\\
      &= (z-z_0) (-b\sin z_0 + O(z-z_0)).
    \end{align*}
    Since
    \begin{align*}
      \sin z_0 &= \frac{\me^{\mi z_0} - \me^{-\mi z_0}}{2\mi}\\
      &= \frac 1{2\mi} \left(\frac{b}{a-\sqrt{a^2-b^2}} -\frac{a-\sqrt{a^2-b^2}}{b} \right)\\
      &= \frac 1 {\mi} \frac{\sqrt{a^2-b^2}}{b},
    \end{align*}
    we have $a + b\cos z = (z-z_0)\left(\mi \sqrt{a^2-b^2} + O(z-z_0)\right)$.

    Now consider a rectangular contour $\gamma_R$ from $0$ to $2\pi$, to $2\pi+R\mi$, to $R\mi$ and then back to $0$.
    There is only one pole inside the region, residue formula gives
    \begin{align*}
      \int_{\gamma_R} \frac{\dd z}{a + b\cos z} & = 2\pi \mi \res_{z_0} \frac 1{a + b\cos z}\\
      &= \frac{2\pi}{\sqrt{a^2-b^2}}.
    \end{align*}
    Note that the two vertical lines cancel each other, we have
    \begin{align*}
      \int_0^{2\pi} \frac{\dd \theta}{a + b\cos \theta} &= \frac{2\pi}{\sqrt{a^2-b^2}}
      + \int_0^{2\pi} \frac{\dd \theta}{a + b \cos(\theta + \mi R)},
    \end{align*}
    since
    \begin{align*}
      \left|\cos (\theta + \mi R)\right| &= \left| \frac{\me^{\mi(\theta + \mi R)}+\me^{-\mi(\theta + \mi R)}}{2} \right|\\
      &\geq \frac 12 (\me^{R} - \me^{-R}),
    \end{align*}
    we have
    \begin{align*}
      &\quad \left|\int_0^{2\pi} \frac{\dd \theta}{a + b \cos(\theta + \mi R)}\right|\\
      &\leq \int_0^{2\pi} \frac{\dd \theta}{(b/2)(\me^{R}-\me^{-R}) - a}\\
      &= \frac{2\pi}{(b/2)(\me^{R}-\me^{-R}) - a} \to 0.
    \end{align*}
    So we have
    \[ \int_0^{2\pi} \frac{\dd \theta}{a + b\cos \theta} = \frac{2\pi}{\sqrt{a^2-b^2}}. \qedhere \]
  \end{proof}
  \setcounter{enumi}{9}
  \item Show that if $a > 0$, then
  \[ \int_0^\infty \frac{\log x}{x^2+a^2}\dd x = \frac{\pi}{2a}\log a. \]
  \begin{proof}
    We can define $\log z$ on $\bbC \smallsetminus \{ \mi y \mid y \leq 0 \}$.
    Consider the integration along the intended semicircle for $\epsilon$ small enough
    and $R$ large enough, the residue formula gives
    \begin{align*}
      &\quad \int_\epsilon^R \frac{(2\log x + \mi \pi)}{x^2+a^2} \dd x
      + \int_{\gamma_R^+} \frac{\log z}{z^2+a^2}\dd z
      + \int_{\gamma_\epsilon^+} \frac{\log z}{z^2+a^2}\dd z\\
      &= 2\pi \mi \res_{\mi a} \frac{\log z}{z^2+a^2}\\
      &= 2\pi \mi \lim_{z\to \mi a} \frac{(z-\mi a)\log z}{z^2+a^2}\\
      &= 2\pi \mi \frac{\log \mi a}{2\mi a}\\
      &= \frac{\pi}{a} \left(\log a + \frac{\pi}{2}\mi\right).
    \end{align*}
    Then we compute the integration on the two semicircles.
    For $\gamma_R^+$ we have
    \begin{align*}
      &\quad \left|\int_{\gamma_R^+} \frac{\log z}{z^2+a^2}\dd z\right|\\
      &= \left|\int_0^\pi \frac{\log (R\me^{\mi \theta})}{R^2\me^{2\mi \theta}+a^2}\dd (R\me^{\mi \theta})\right|\\
      &\leq \int_0^\pi \frac{\left|\log R + \mi \theta\right|}{R^2-a^2} \cdot R \dd \theta\\
      &\leq \frac{\pi R(\log R + \pi)}{R^2-a^2}\to 0.
    \end{align*}
    For $\gamma_\epsilon^+$ we have
    \begin{align*}
      &\quad \left|\int_{\gamma_\epsilon^+} \frac{\log z}{z^2+a^2}\dd z\right|\\
      &= \left|\int_0^\pi \frac{\log (\epsilon\me^{\mi \theta})}{\epsilon^2\me^{2\mi \theta}+a^2}\dd (\epsilon\me^{\mi \theta})\right|\\
      &\leq \int_0^\pi \left|\frac{\epsilon \mi (\log \epsilon + \mi \theta)}{\epsilon^2\me^{2\mi \theta}+a^2}\right| \dd \theta\\
      &\leq \frac{\pi\epsilon (\log \epsilon + \pi)}{a^2-\epsilon^2}\to 0.
    \end{align*}
    So we have
    \[ \int_0^\infty \frac{(2\log x + \mi \pi)}{x^2+a^2} \dd x = \frac{\pi}{a} \left(\log a + \frac{\pi}{2}\mi\right). \]
    Take the real part, we have 
    \[ \int_0^\infty \frac{\log x}{x^2+a^2} \dd x = \frac{\pi \log a}{2a}. \qedhere \]
  \end{proof}
  \setcounter{enumi}{11}
  \item Suppose $u$ is not an integer. Prove that
  \[ \sum_{n=-\infty}^\infty \frac 1{(u+n)^2} = \frac{\pi^2}{(\sin \pi u)^2} \]
  by integrating
  \[ f(z) = \frac{\pi \cot \pi z}{(u+z)^2} \]
  over the circle $|z| = R_N = N + 1/2$ ($N$ integral, $N \geq |u|$), adding the
  residues of $f$ inside the circle, and letting $N$ tend to infinity.
  \begin{proof}
    We first analyze the poles of $\pi \cot \pi z / (u+z)^2$, at $z=n$,
    we have
    \begin{align*}
      &\quad \res_n \frac{\pi \cot \pi z}{(u+z)^2}\\
      &= \frac{\pi \cot \pi n}{(u+n)^2} \res_n \frac 1{\sin \pi z}\\
      &= \frac{\pi (-1)^n}{(u+n)^2} \cdot (-1)^n \frac 1 \pi\\
      &= \frac 1{(u+n)^2}.
    \end{align*}
    At $z=-u$, we have
    \begin{align*}
      &\quad \res_{-u} \frac{\pi \cot \pi z}{(u+z)^2}\\
      &= \lim_{z\to -u} \frac{\dd}{\dd z} \pi \cot \pi z\\
      &= \lim_{z\to -u} -\frac{\pi^2} {\sin^2 \pi z}\\
      &= -\frac{\pi^2} {\sin^2 \pi u}.
    \end{align*}
    Then integrate along the circle $C_{R_N}$, the residue formula gives
    \[ \int_{C_{R_N}} f(z)\dd z = 2\pi \mi \left(\sum_{n=-N}^N \frac{1}{(u+n)^2} - \frac{\pi^2}{\sin^2\pi u}\right). \]
    Note that
    \begin{align*}
      \left|\frac{\pi \cot \pi z}{(u+z)^2}\right| &= \frac{\pi}{|u+z|^2} \left|
        \frac{\me^{\mi \pi z}+\me^{-\mi \pi z}}{\me^{\mi \pi z}-\me^{-\mi \pi z}} \right|\\
        &= \frac{\pi}{|u+z|^2} \left|
          1+\frac{2}{\me^{2\mi \pi z}-1} \right|,
    \end{align*}
    note that $|\me^{2\mi \pi z} - 1| < \epsilon$ corresponds to a small neighborhood
    around each $z=n$, 
    so $\left|\frac{\pi \cot \pi z}{(u+z)^2}\right| \leq B$ for some constant $B$ when $N$ large enough.
    The integration converges to $0$. We have
    \[ \sum_{n=-N}^N \frac{1}{(u+n)^2} - \frac{\pi^2}{\sin^2\pi u} \to 0. \]
    That is,
    \[ \sum_{n=-\infty}^\infty \frac{1}{(u+n)^2} = \frac{\pi^2}{\sin^2\pi u}. \qedhere \]
  \end{proof}
  \item Suppose $f(z)$ is holomorphic in a punctured disc $D_r(z_0) - \{z_0\}$. Suppose also that
  \[ |f(z_0)| \leq A |z-z_0|^{-1+\varepsilon} \]
  for some $\varepsilon>0$, and all $z$ near $z_0$. Show that the singularity of $f$ at $z_0$ is removable.
  \begin{proof}
    The proof is basically repeating Riemann's theorem. We first choose a circle $C$
    inside the disc and containing $z_0$, if
    \[ f(z) = \frac 1{2\pi \mi} \int_C \frac{f(\zeta)}{\zeta-z}\dd \zeta \]
    holds for all $z \neq z_0$ inside $C$, then we can extend to $z_0$ by continuity.
    Let $\gamma_\epsilon$ be connecting the circle and two clockwise $\epsilon$-circle around $z$ and $z_0$
    respectively, we have
    \[ \int_\gamma \frac{f(\zeta)}{\zeta-z}\dd \zeta = 0, \]
    so we have
    \begin{align*}
      \frac 1{2\pi \mi} \int_C \frac{f(\zeta)}{\zeta-z}\dd \zeta &=
      \frac 1{2\pi \mi} \left( \int_{C_\epsilon(z_0)} \frac{f(\zeta)}{\zeta-z}\dd \zeta
      +\int_{C_\epsilon(z)} \frac{f(\zeta)}{\zeta-z}\dd \zeta \right) \\
      &= f(z) + \frac 1{2\pi \mi}\int_{C_\epsilon(z_0)} \frac{f(\zeta)}{\zeta-z}\dd \zeta.
    \end{align*}
    Note that when $\epsilon$ small enough, we have
    \[ \left|\frac{f(\zeta)}{\zeta-z}\dd \zeta\right| \leq \frac{A|\zeta - z_0|^{-1+\varepsilon}}{|z-z_0|-\epsilon}, \]
    we have
    \begin{align*}
      &\quad \left|\frac 1{2\pi \mi}\int_{C_\epsilon(z_0)} \frac{f(\zeta)}{\zeta-z}\dd \zeta\right|\\
      &\leq \frac{A \epsilon^{\varepsilon}}{|z-z_0|- \epsilon} \to 0.
    \end{align*}
    So we have the equality holds.
  \end{proof}
  \item Prove that all entire functions that are also injective take the form
  $f(z)=az+b$ with $a, b \in \bbC$, and $a\neq 0$.
  \begin{proof}
    If $f$ is meromorphic over the extened complex plane, then $f$ is a rational function.
    Since $f$ is entire, it must be a polynomial. Since $f$ is injective, it must have the
    form $az+b$ where $a\neq 0$.

    Otherwise, $f(1/z)$ has an essential singularity at $z=0$. Casorati--Weierstrass theorem
    tells that the image of $f(1/z)$ is dense around $z=0$, say, $D_{1/2}(0) \smallsetminus\{0\}$.
    The open mapping theorem tells that the image of $D_{1/2}(1)$ should be open,
    so their image has some intersecion $f(1/z_1) = f(1/z_2)$ where $|z_1|, |z_2-1| < 1/2$.
    $f$ cannot be injective.
  \end{proof}
  \setcounter{enumi}{15}
  \item Suppose $f$ and $g$ are holomorphic in a region containing the disc $|z| \leq 1$. Suppose that
  $f$ has a simple zero at $z = 0$ and vanishes nowhere else in $|z| \leq 1$. Let
  \[ f_\epsilon(z) = f(z) + \epsilon g(z). \]
  Show that if $\epsilon$ is sufficiently small, then
  \begin{enumerate}
    \item $f_\epsilon(z)$ has a unique zero in $|z| \leq 1$, and
    \begin{proof}
      When
      \[ \epsilon < \inf_{|z|=1} \frac{|f(z)|}{|g(z)|}, \]
      Rouch\'e's theorem applies.
    \end{proof}
    \item if $z_\epsilon$ is this zero, the mapping $\epsilon\to z_\epsilon$ is continuous.
    \begin{proof}
      We first show that if $f$ has exactly one zero $|z_0|<1$ inside the unit circle $\gamma$ and vanishes
      nowhere else in $|z|\leq 1$, then we have
      \[ z_0 = \frac 1{2\pi \mi} \int_\gamma \frac{z f'(z)}{f(z)} \dd z. \]
      Since $z_0$ is the only zero, we have $f(z) = g(z)(z-z_0)$ where $g(z)$ is nonzero and
      holomorphic in $|z|\leq 1$, we have
      \begin{align*}
        \frac{zf'(z)}{f(z)} &= z\left( \frac{1}{z-z_0} + \frac{g'(z)}{g(z)} \right)\\
        &= 1 + \frac{z_0}{z-z_0} + \frac{zg'(z)}{g(z)}.
      \end{align*}
      Since $1 + zg'(z)/g(z)$ is holomorphic, we have
      \begin{align*}
        &\quad \frac 1{2\pi \mi} \int_\gamma \frac{zf'(z)}{f(z)}\dd z\\
        &= \frac 1{2\pi \mi} \frac{z_0}{z-z_0}\dd z\\
        &= z_0.
      \end{align*}
      Therefore, we have
      \[ z_\epsilon = \frac 1{2\pi \mi} \int_\gamma \frac{z(f'(z)+\epsilon g'(z))}{f(z)+\epsilon g(z)} \dd z, \]
      which is clearly continuous respect to $\epsilon$.
    \end{proof}
  \end{enumerate}
  \setcounter{enumi}{19}
  \item This exercise shows how the mean square convergence dominates the uniform convergence of analytic
  functions. If $U$ is an open subset of $\bbC$ we use the notation
  \[ \norm{f}_{L^2(U)} = \left(\int_U |f(z)|^2 \dd x\dd y\right)^{1/2} \]
  for the mean square norm, and
  \[ \norm{f}_{L^\infty(U)} = \sup_{z\in U} |f(z)| \]
  for the sup norm.
  \begin{enumerate}
    \item If $f$ is holomorphic in a neighborhood of the disc $D_r(z_0)$, show that for any $0<s<r$
    there exists a constant $C>0$ (which depends on $s$ and $r$) such that
    \[ \norm{f}_{L^\infty(D_s(z_0))} \leq C \norm{f}_{L^2(D_r(z_0))}. \]
    \begin{proof}
      By the maximum modulus principle, the sup is taken on some $|z_s|=s$. For $\rho < r-s$, by the mean-value
      property, we have
      \begin{align*}
        |f(z_s)| &= \left|\frac 1{2\pi} \int_0^{2\pi} f(z_s + \rho \me^{\mi \theta})\dd \theta \right|\\
        &\leq \frac 1{2\pi} \int_0^{2\pi} |f(z_s+\rho \me^{\mi \theta})| \dd \theta.
      \end{align*}
      Therefore, we have
      \begin{align*}
        &\quad \int_{|z-z_s|< r-s} |f(z)| \dd x\dd y\\
        &= \int_{0}^{r-s} \rho \left(\int_0^{2\pi}|f(z_s+\rho^{\mi \theta})| \dd \theta\right)\dd \rho\\
        &\geq \int_{0}^{r-s} \rho \cdot 2\pi |f(z_s)|\dd \rho\\
        &= \pi (r-s)^2 |f(z_s)|.
      \end{align*}
      At last, by Cauchy--Schwarz inequality, we have
      \begin{align*}
        &\quad \int_{|z-z_s|< r-s} |f(z)| \dd x\dd y\\
        &\leq \left(\int_{|z-z_s|< r-s} 1 \dd x\dd y\right)^{1/2}\left(\int_{|z-z_s|< r-s} |f(z)|^2 \dd x\dd y\right)^{1/2}\\
        &\leq \sqrt{\pi} (r-s) \norm{f}_{L^2(D_r(z_0))},
      \end{align*}
      so we have
      \[ \norm{f}_{L^\infty(D_s(z_0))}\leq \frac{1}{\sqrt{\pi}(r-s)} \norm{f}_{L^2(D_r(z_0))}.\qedhere \]
    \end{proof}
    \item Prove that if $\{f_n\}$ is a Cauchy sequence of holomorphic functions in the mean square
    norm $\norm{\cdot}_{L^2(U)}$, then the sequence $\{f_n\}$ converges uniformly on every compact
    subset of $U$ to a holomorphic function.
    \begin{proof}
      By compactness, we can find finite discs $D_{r}(z) \subset U$ such that
      $D_{r/2}(z)$ covers the subset, say, $C$. Therefore, the $L^2$ convergence gives $L^\infty$
      convergence, then $f_n$ has a uniform limit over $C$. Therefore, theorem 5.2 applies.
    \end{proof}
  \end{enumerate}
\end{enumerate}

\section{Problems}

\begin{enumerate}
  \item Consider a holomorphic map on the unit disc $f \colon \bbD \to \bbC$ which satisfies $f(0) = 0$.
  By the open mapping theorem, the image $f(\bbD)$ contains a small disc centered at the origin.
  We then ask: does there exist $r > 0$ such that for all $f \colon \bbD \to \bbC$ with
  $f(0) = 0$, we have $D_r(0) \subset f(\bbD)$?
  \begin{enumerate}
    \item Show that with no further restrictions on $f$, no such $r$ exists. It suffices to find a
    sequence of functions $\{f_n\}$ holomorphic in $\bbD$ such that $1/n \notin f(\bbD)$. Compute
    $f_n'(0)$, and discuss.
    \begin{proof}
      Let $f_n(z) = z/n$, since $|f_n(z)| < 1/n$, we have $1/n\notin f(\bbD)$.
      In our construction, $f'_n(0)=1/n \to 0$.
    \end{proof}
    \item Assume in addition that $f$ also satisfies $f'(0) = 1$. Show that despite this new assumption,
    there exists no $r > 0$ satisfying the desired condition.
    \begin{proof}
      Let $f_\epsilon(z) = \epsilon(\me^{z/\epsilon}-1)$, the solution of $f_\epsilon(z)=-1/n$
      is
      \[ z = \epsilon \left( 2\pi m \mi + \log \left(1 - \frac 1{n\epsilon}\right) \right), \quad m\in\mathbb Z. \]
      The solution with smallest modulus is $m=0$, we have
      \[ z = \epsilon \log \left(1 - \frac 1{n\epsilon}\right), \]
      we have $|z|\to \infty$ as $\epsilon\to 0$. So we have $-1/n\notin f_\epsilon(\bbD)$ for $\epsilon$
      small enough.
    \end{proof}
    The Koebe--Bieberbach theorem states that if in addition to $f(0) = 0$ and $f'(0) = 1$ we also
    assume that $f$ is injective, then such an $r$ exists and the best possible value is $r = 1/4$.
    \item As a first step, show that if $h(z) = \frac 1 z + c_0 + c_1 z + c_2z^2 + \cdots$ is analytic and
    injective for $0 < |z| < 1$, then $\sum_{n=1}^\infty n |c_n|^2 \leq 1$. \label{area}
    \begin{proof}
      Fix $\rho\in (0, 1)$, consider the image of $D_\rho(0)$ through $1/h(z)$. Note that changing $c_0$ has no effect,
      we assume $h(z)$ has no zero at $|z| \leq \rho$. Therefore, $1/h(z)$ is holomorphic
      on an open set containing $\overline{D_\rho(0)}$, the image should be compact and simply connected,
      and it contains $0$. Therefore, the complement of $h(D_\rho(0))$ should be compact and simply connected.
      We can compute its area by Green's theorem, let $f(\theta) = h(\rho \me^{\mi \theta})$, we have
      \begin{align*}
        |A| &= \frac 12\left| \int_0^{2\pi} (\Re f(\theta) \Im f'(\theta)
        - \Im f(\theta) \Re f'(\theta)) \dd \theta \right|\\
        &= \frac 12 \left| \int_0^{2\pi} (\Re h(\rho \me^{\mi\theta}) \Im \mi \me^{\mi \theta} h'(\rho \me^{\mi\theta})
        - \Im h(\rho \me^{\mi\theta}) \Re \mi \me^{\mi \theta} h'(\rho \me^{\mi\theta})) \rho\dd \theta \right|\\
        &= \frac 12\left| \int_0^{2\pi} (\Re h(\rho \me^{\mi\theta}) \Re \me^{\mi \theta} h'(\rho \me^{\mi\theta})
        + \Im h(\rho \me^{\mi\theta}) \Im \me^{\mi \theta} h'(\rho \me^{\mi\theta})) \rho\dd \theta \right|\\
        &= \frac 12\left| \Re \int_0^{2\pi} h(\rho\me^{\mi \theta}) \overline{\rho \me^{\mi \theta} h'(\rho\me^{\mi\theta})}\dd \theta \right|,
      \end{align*}
      since $\rho<1$, the series is absolute converging, we have
      \begin{align*}
        &= \frac 12 \left| \Re \int_0^{2\pi} \sum_{n, m\geq -1}
        c_n \rho^n \me^{n\mi \theta}\cdot m\overline{c_m} \rho^{m} \me^{-m\mi \theta} \dd \theta \right|\\
        &= \frac 12 \left| \Re \sum_{n, m\geq -1}
        c_n \rho^n \cdot m\overline{c_m} \rho^{m}\int_0^{2\pi}  \me^{(n-m)\mi \theta}  \dd \theta \right|\\
        &= \frac 12 \left| \Re \sum_{n\geq -1}
        c_n \rho^n \cdot n\overline{c_{n}} \rho^{n} \cdot 2\pi \right|\\
        &= \pi \left| \rho^{-2} - \sum_{n=1}^\infty n |c_n|^2 \rho^{2n} \right|\\
        &= \pi \left(\rho^{-2} - \sum_{n=1}^\infty n |c_n|^2 \rho^{2n}\right).
      \end{align*}
      Since $|A|\geq 0$, we must have
      \[ \sum_{n=1}^\infty n |c_n|^2 \rho^{2n} \leq \rho^{-2}. \]
      Take $\rho\to 1$, we have
      \[ \sum_{n=1}^\infty n |c_n|^2 \leq 1. \]
    \end{proof}
    \item If $f(z)=z + a_2 z^2 + \cdots$ satisfies the hypotheses of the theorem, show
    that there exists another function $g$ satisfying the hypotheses of the theorem such
    that $g^2(z) = f(z^2)$.
    \begin{proof}
      By the hypotheses, $0$ is the only zero of $f(z)$. Thus $f(z)/z$ is nowhere vanishing.
      Since the disc is simply connected, we can have holomorphic function $h$ such that
      $h(0)=0$, and $f(z)/z=\me^{h(z)}$. Then we take $\psi(z) = \me^{h(z)/2}$, we have
      $\psi(z)^2 = f(z)/z$. Then let $g(z) = z\psi (z^2)$, we have $g(z)^2 = f(z^2)$.

      Clearly we have $g(0) = f(0)=0$ and $g'(0) = \psi(0) = 1$. We prove that $g$ is injective.
      For $z_1 \neq z_2$, if $z_1^2 \neq z_2^2$, we have $f(z_1^2)\neq f(z_2^2)$, thus $g(z_1)\neq g(z_2)$.
      Otherwise, we only need to prove $g(z)\neq g(-z)$ for $z\neq 0$, this follows from
      $g(z) = z \psi(z^2)$, since $\psi$ is nonvanishing, we have $g(z)=-g(-z)\neq 0$.
    \end{proof}
    \item With the notation of the previous
    part, show that $|a_2|\leq 2$, and that equality holds if and only if
    \begin{center}
      $\displaystyle f(z) = \frac{z}{(1-\me^{\mi \theta}z)^2}$\quad for some $\theta\in\bbR$.
    \end{center}
    \begin{proof}
      Not hard to see that $g(z) = z + \frac{a_2}{2}z^3 + O(z^4)$, then
      \[ h(z) = \frac 1{g(z)} = z^{-1} - \frac{a_2}{2}z + O(z^2) \]
      is analytic and injective for $0<|z|<1$. Then \ref{area} applies, we have $|a_2|\leq 2$.
      When the equality holds, we must have $h(z) = z^{-1} - \me^{\mi \theta} z$, thus
      $g(z) = z / (1 - \me^{\mi \theta}z^2)$, $f(z) = z/(1-\me^{\mi\theta} z)^2$.
      To show such $f$ is injective, we only need to show for $\theta=0$.
      Suppose $f(z)=y$, when $y=0$ the only solution is clearly $0$, otherwise we have
      \[ z^2 - \left(2 + \frac 1 y\right)z + 1 = 0. \]
      So the two solutions satisfy $z_1z_2 = 1$, there is at most one solution lying in $D_1(0)$.
    \end{proof}
    \item If $h(z)= \frac1z +c_0 +c_1z+c_2z^2 +\cdots$ is injective on $\bbD$ and avoids the values
    $z_1$ and $z_2$, show that $|z_1-z_2|\leq 4$. \label{havoid}
    \begin{proof}
      Consider $f(z) = 1/(h(z) - z_j)$, we have
      \begin{align*}
        f(z) &= \frac{1}{1/z + (c_0 -z_j) + O(z)}\\
        &= z + (z_j - c_0)z^2 + O(z^3),
      \end{align*}
      since $h(z)-z_j$ never vanishes, we have $h(z)$ is holomorphic and injective on $\bbD$,
      thus $|z_j - c_0|\leq 2$. Thus $|z_1-z_2|\leq 4$.
    \end{proof}
    \item Complete the proof of the theorem.
    \begin{proof}
      If $f(z) = z + O(z^2)$ is injective and holomorphic over $\bbD$,
      suppose $f$ avoids $w$, then $h(z)=1/f(z)$ has the form in \ref{havoid}
      and avoids $0$ and $1/w$, so we have $|1/w|\leq 4$, thus $|w|\geq 4$.
      We can see that $D_{1/4}(0)$ always lies in the image of $f$.

      To achieve $1/4$, consider $f(z) = z/(1-z)^2$ and $f(z) = -1/4$, the only solution is $z=-1$.
    \end{proof}
    \setcounter{enumi}{2}
    \item If $f(z)$ is holomorphic in the deleted neighborhood $\{0 < |z - z_0| < r\}$ and has
    a pole of order $k$ at $z_0$, then we can write
    \[ f(z) = \frac{a_{-k}}{(z-z_0)^k} + \cdots + \frac{a_{-1}}{z-z_0} + g(z) \]
    where $g$ is holomorphic in the disc $\{|z - z_0| < r\}$. There is a generalization of this
    expansion that holds even if $z_0$ is an essential singularity. This is a special case of
    the \textbf{Laurent series expansion}, which is valid in an even more general setting.

    Let $f$ be holomorphic in a region containing the annulus $\{z : r_1 \leq |z - z_0| \leq r_2\}$
    where $0 < r_1 < r_2$. Then,
    \[ f(z) = \sum_{n=-\infty}^{\infty} a_n (z-z_0)^n \]
    where the series converges absolutely in the interior of the annulus. To prove this,
    it suffices to write
    \[ f(z) = \frac 1{2\pi \mi} \int_{C_{r_2}} \frac{f(\zeta)}{\zeta-z}\dd \zeta
    - \frac 1{2\pi \mi}\int_{C_{r_1}} \frac{f(\zeta)}{\zeta-z}\dd \zeta \]
    when $r_1 < |z - z_0| < r_2$, and argue as in the proof of Theorem 4.4, Chapter 2.
    Here $C_{r_1}$ and $C_{r_2}$ are the circles bounding the annulus.
    \begin{proof}
      The above formula is true because we can add a line connecting the two circles,
      and $z$ is contained in the interior.

      Then for the $r_2$ part, we use the convergence
      \[ \frac 1{\zeta - z} = \frac 1{\zeta - z_0 - (z - z_0)} = \frac 1{\zeta - z_0}\frac 1{1 - \frac{z-z_0}{\zeta-z_0}}, \]
      we have
      \begin{align*}
        &\quad \frac 1{2\pi \mi} \int_{C_{r_2}} \frac{f(\zeta)}{\zeta-z}\dd \zeta\\
        &= \frac 1{2\pi \mi} \int_{C_{r_2}} \sum_{n=0}^\infty \frac {f(\zeta)(z-z_0)^{n}}{(\zeta - z_0)^{n+1}} \dd \zeta\\
        &= \sum_{n=0}^\infty \left(\frac 1{2\pi \mi} \int_{C_{r_2}} \frac {f(\zeta)}{(\zeta - z_0)^{n+1}} \dd \zeta\right) (z-z_0)^{n},
      \end{align*}
      and for the $r_1$ part, we use the convergence
      \[ \frac 1{\zeta - z} = \frac 1{z-z_0} \frac 1{\frac{\zeta - z_0}{z-z_0} - 1}, \]
      we have
      \begin{align*}
        &\quad \frac 1{2\pi \mi}\int_{C_{r_1}} \frac{f(\zeta)}{\zeta-z}\dd \zeta\\
        &= -\frac 1{2\pi \mi}\int_{C_{r_1}} \sum_{n=0}^\infty \frac{f(\zeta) (\zeta-z_0)^n}{(z-z_0)^{n+1}} \dd \zeta\\
        &= -\sum_{n=0}^\infty \left( \frac 1{2\pi \mi}\int_{C_{r_1}}f(\zeta) (\zeta-z_0)^n \dd \zeta\right) \frac{1}{(z-z_0)^{n+1}}\\
        &= -\sum_{n\leq -1} \left( \frac 1{2\pi \mi}\int_{C_{r_1}} \frac{f(\zeta)}{(\zeta-z_0)^{n+1}} \dd \zeta\right) (z-z_0)^{n}.
      \end{align*}
      In conclusion, the sequence $a_n$ satisfies
      \begin{center}
        $\displaystyle a_n = \frac 1{2\pi \mi}\int_{C_{r_2}} \frac{f(\zeta)}{(\zeta-z_0)^{n+1}} \dd \zeta$
        \quad for $n\geq 0$,
      \end{center}
      and
      \begin{center}
        $\displaystyle \frac 1{2\pi \mi}\int_{C_{r_1}} \frac{f(\zeta)}{(\zeta-z_0)^{n+1}} \dd \zeta$\quad for $n\leq -1$.
      \end{center}
      The convergence is not hard to see.
    \end{proof}
  \end{enumerate}
\end{enumerate}

\end{document}
