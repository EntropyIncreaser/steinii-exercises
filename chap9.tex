%!TEX encoding = UTF-8 Unicode
\documentclass[11pt]{report}
\usepackage[margin = 1in]{geometry}

\usepackage{amsmath}
\usepackage{amssymb}
\usepackage{mleftright}
\usepackage{enumitem}
\usepackage{textcomp, gensymb}
\usepackage{tikz, tikz-cd}
\usepackage{bbding}
\usepackage{tabularx}
\usepackage{nicematrix}
\usepackage{extarrows}
\usepackage{lastpage}
\usepackage{hyperref}

\usepackage{fancyhdr}  % Header and Footer formatting

\usetikzlibrary{decorations.markings}

\usepackage{multicol}

\hypersetup{colorlinks = true,
            linkcolor = blue,
            citecolor = red,
            urlcolor = teal}

\pagestyle{fancy}
\renewcommand{\headrulewidth}{0.4pt}
\renewcommand{\footrulewidth}{0.4pt}
\setlength{\headheight}{18pt}

% Header and Footer Information
\lhead{\small\emph{Baitian Li}}
\chead{}
\rhead{\textsc{Complex Analysis}}
\lfoot{\today}
\cfoot{}
\rfoot{\thepage\ of \pageref{LastPage}}  % Counts the pages.

\makeatletter % This provides a total page count as \ref{NumPages}
\AtEndDocument{\immediate\write\@auxout{\string\newlabel{NumPages}{{\thepage}}}}
\makeatother

%\lineskiplimit=-\maxdimen\relax

\usepackage{amsthm}  % This will create the Problem environment
\newtheorem*{lemma}{Lemma}
\newtheorem*{theorem}{Theorem}
\newtheoremstyle{mythm}%
    {12pt}{}%
    {\sffamily}{}%
    {\bfseries \sffamily}{.}%
    {.5em}%
    {\thmname{#1}\thmnumber{ #2}\thmnote{ #3}}

\theoremstyle{mythm}

\expandafter\let\expandafter\oldproof\csname\string\proof\endcsname
\let\oldendproof\endproof
\renewenvironment{proof}[1][\proofname]{%
  \oldproof[\normalfont \bfseries #1]%
}{\oldendproof}

\renewcommand*{\proofname}{Proof}

\newtheoremstyle{myans}%
    {12pt}{}%
    {}{}%
    {\bfseries}{.}%
    {.5em}%
    {\thmname{#1}\thmnumber{ #2}\thmnote{ #3}}

\theoremstyle{myans}
\newtheorem*{answer}{Answer}

\setlist[enumerate]{noitemsep, topsep = 0.2em}
\setlist[enumerate]{label = {\bfseries \arabic*.}}
\setlist[enumerate, 2]{label = {(\alph*)}}
\setlist[description]{topsep = 0.2em, listparindent = \parindent, font = \normalfont,
  itemsep = 0em}

\newcommand{\mi}{\mathrm{i}}
\newcommand{\me}{\mathrm{e}}
\newcommand{\dd}{\mathop{}\!\mathrm{d}}

\newcommand{\bbC}{\mathbb{C}}
\newcommand{\bbD}{\mathbb{D}}
\newcommand{\bbR}{\mathbb{R}}

\newcommand{\norm}[1]{\|#1\|}

\renewcommand{\Re}{\operatorname{Re}}
\renewcommand{\Im}{\operatorname{Im}}

\DeclareMathOperator{\res}{res}

\allowdisplaybreaks

\begin{document}

\setcounter{chapter}{8}
\chapter{An Introduction to Elliptic Functions}

\section{Exercises}

\begin{enumerate}
  \setcounter{enumi}{3}
  \item By rearranging the series
  \[ \frac 1{z^2} + \sum_{\omega \in \Lambda^*} \left[\frac1{(z+\omega)^2} - \frac1{\omega^2}\right], \]
  show directly, without differentiation, that $\wp(z + \omega) = \wp(z)$ whenever $\omega \in \Lambda$.
  \begin{proof}
    By the absolute convergence, we can write $\wp(z)$ into the form
    \[ \wp(z) = \frac 1{z^2} + \lim_{R\to\infty} \sum_{0<|\omega|<R} \left[\frac1{(z+\omega)^2} - \frac1{\omega^2}\right], \]
    therefore, for $\lambda \in \Lambda$ we have
    \[ \wp(z) - \wp(z + \lambda) = \lim_{R\to\infty} \sum_{|\omega|<R} \left[\frac1{(z+\omega)^2} - \frac1{(z+\lambda+\omega)^2}\right], \]
    we have the terms $|\omega| < R-\lambda$ are all cancelled, so the inner summation is $O(1/R^2)\cdot O(R) = O(1/R)$
    tends to $0$. Thus $\pi(z)= \pi(z+\lambda)$.
  \end{proof}
  \setcounter{enumi}{5}
  \item Prove that $\wp''$ is a quadratic polynomial in $\wp$.
  \begin{proof}
    Consider the expansion of $\wp$, we have
    \begin{align*}
      \wp(z) &= z^{-2} + 3E_4 z^2 + \cdots,\\
      \wp''(z) &= 6z^{-4} + 6 E_4 + \cdots,\\
      \wp(z)^2 &= z^{-4} + 6E_4 + \cdots.
    \end{align*}
    Then we have $\wp''(z) - 6\wp(z)^2 + 30E_4$ is entire and zero at $0$. Theorem 1.2 applies, we know that
    it is identically zero. So we have
    \[ \wp''(z) = 6\wp(z)^2 - 30E_4. \qedhere \]
  \end{proof}
  \setcounter{enumi}{7}
  \item Let
  \[ E_4(\tau) = \sum_{(n,m)\neq (0,0)} \frac 1{(n+m\tau)^4} \]
  be the Eisenstein series of order $4$.
  \begin{enumerate}
    \item Show that $E_4(\tau) \to \pi^4/45$ as $\Im(\tau) \to \infty$.
    \begin{proof}
      Apply theorem 2.5, we have
      \[ E_4(\tau) = 2\zeta(4) + \frac{1}{3}\sum_{r=1}^\infty \sigma_{3}(r)\me^{2\pi \mi \tau r}. \]
      Since the summation is absolute convergent, when $\Im(\tau) \to \infty$, it tends to $0$, so we have
      \[ E_4(\tau) \to 2\zeta(4) = \frac{\pi^4}{45}. \qedhere \]
    \end{proof}
    \item More precisely,
    \begin{center}
      $\displaystyle\left|E_4(\tau) - \frac{\pi^4}{45}\right| \leq c \me^{-2\pi t}$ \quad if
      $\tau = x + \mi t$ and $t\geq 1$.
    \end{center}
    \begin{proof}
      \begin{align*}
        &\quad \displaystyle\left|E_4(\tau) - \frac{\pi^4}{45}\right|\\
        &= \frac 13 \left| \sum_{r=1}^\infty \sigma_3(r) \me^{2\pi \mi \tau r} \right|\\
        &\leq \frac 13 \sum_{r=1}^\infty r^4 \me^{-2\pi r t},
      \end{align*}
      which is clearly bounded by some $c\me^{-2\pi t}$ when $t\geq 1$.
    \end{proof}
    \item Deduce that
    \begin{center}
      $\displaystyle \left| E_4(\tau) - \tau^{-4} \frac{\pi^4}{45}\right| \leq ct^{-4}\me^{-2\pi /t}$
      \quad if $\tau = \mi t$ and $0 < t \leq 1$.
    \end{center}
    \begin{proof}
      Recall that $E_4(\tau) = \tau^{-4} E_4(-1/\tau)$, we have
      \[ \left| E_4(\tau) - \tau^{-4} \frac{\pi^4}{45}\right| = |\tau|^{-4} \left| E_4(-1/\tau) - \frac{\pi^4}{45}\right|, \]
      since $-1/\tau = \mi / t$, we have $1/t \geq 1$, thus
      \[ \leq ct^{-4} \me^{-2\pi/t}. \qedhere \]
    \end{proof}
  \end{enumerate}
\end{enumerate}

\section{Problems}

\begin{enumerate}
  \setcounter{enumi}{1}
  \item Show that
  \[ \wp(z) = c + \pi^2 \sum_{m=-\infty}^{\infty} \frac 1{\sin^2((z + m\tau) \pi)} \]
  where $c$ is an appropriate constant. In fact, by part (b) of the previous problem $c = -F(\tau)$.
  \begin{proof}
    Let $f(z) = \pi^2 \sum_{m=-\infty}^{\infty} \frac 1{\sin^2((z + m\tau) \pi)}$.

    Since $1/\sin z$ has an exponential decay when $\Im z\to \infty$, we know the summation has
    absolute convergence on a compact set.

    From the periodicity of $\sin$, we know that $f(z)$ has periodicity $1$. From the summation,
    we know that $f(z)$ has periodicity $\tau$. Note that $\wp(z) - f(z)$ is entire,
    it is a constant.
  \end{proof}
  \item Suppose $\Omega$ is a simply connected domain that excludes the three roots of the
  polynomial $4z^3 - g_2 z - g_3$. For $\omega_0 \in \Omega$ and $\omega_0$ fixed, define the function
  $I$ on $\Omega$ by
  \[ I(\omega) = \int_{\omega_0}^\omega \frac{\dd z}{\sqrt{4z^3 - g_2 z - g_3}} \quad \omega \in \Omega. \]
  Then the function $I$ has an inverse given by $\wp(z + \alpha)$ for some constant $\alpha$; that is,
  \[ I(\wp(z+\alpha)) = z \]
  for appropriate $\alpha$.
  \begin{proof}
    Consider the derivative of $I(\wp(z))$, we have
    \begin{align*}
      (I(\wp(z)))' &= I'(\wp(z)) \wp'(z)\\
      &= \frac{\wp'}{\sqrt{4 \wp^3 - g_2\wp - g_3}},
    \end{align*}
    since $(\wp')^2 = 4\wp^3 - g_2\wp - g_3$, we have $(I(\wp(z)))' = \pm 1$.
    If $\sqrt{\cdot}$ takes the correct branch, we will have $I(\wp(z))' = 1$, thus
    $I(\wp(z))' = z - \alpha$ for some $\alpha$.
  \end{proof}
\end{enumerate}

\end{document}
