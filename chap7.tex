%!TEX encoding = UTF-8 Unicode
\documentclass[11pt]{report}
\usepackage[margin = 1in]{geometry}

\usepackage{amsmath}
\usepackage{amssymb}
\usepackage{mleftright}
\usepackage{enumitem}
\usepackage{textcomp, gensymb}
\usepackage{tikz, tikz-cd}
\usepackage{bbding}
\usepackage{tabularx}
\usepackage{nicematrix}
\usepackage{extarrows}
\usepackage{lastpage}
\usepackage{hyperref}

\usepackage{fancyhdr}  % Header and Footer formatting

\usetikzlibrary{decorations.markings}

\usepackage{multicol}

\hypersetup{colorlinks = true,
            linkcolor = blue,
            citecolor = red,
            urlcolor = teal}

\pagestyle{fancy}
\renewcommand{\headrulewidth}{0.4pt}
\renewcommand{\footrulewidth}{0.4pt}
\setlength{\headheight}{18pt}

% Header and Footer Information
\lhead{\small\emph{Baitian Li}}
\chead{}
\rhead{\textsc{Complex Analysis}}
\lfoot{\today}
\cfoot{}
\rfoot{\thepage\ of \pageref{LastPage}}  % Counts the pages.

\makeatletter % This provides a total page count as \ref{NumPages}
\AtEndDocument{\immediate\write\@auxout{\string\newlabel{NumPages}{{\thepage}}}}
\makeatother

%\lineskiplimit=-\maxdimen\relax

\usepackage{amsthm}  % This will create the Problem environment
\newtheorem*{lemma}{Lemma}
\newtheorem*{theorem}{Theorem}
\newtheoremstyle{mythm}%
    {12pt}{}%
    {\sffamily}{}%
    {\bfseries \sffamily}{.}%
    {.5em}%
    {\thmname{#1}\thmnumber{ #2}\thmnote{ #3}}

\theoremstyle{mythm}

\expandafter\let\expandafter\oldproof\csname\string\proof\endcsname
\let\oldendproof\endproof
\renewenvironment{proof}[1][\proofname]{%
  \oldproof[\normalfont \bfseries #1]%
}{\oldendproof}

\renewcommand*{\proofname}{Proof}

\newtheoremstyle{myans}%
    {12pt}{}%
    {}{}%
    {\bfseries}{.}%
    {.5em}%
    {\thmname{#1}\thmnumber{ #2}\thmnote{ #3}}

\theoremstyle{myans}
\newtheorem*{answer}{Answer}

\setlist[enumerate]{noitemsep, topsep = 0.2em}
\setlist[enumerate]{label = {\bfseries \arabic*.}}
\setlist[enumerate, 2]{label = {(\alph*)}}
\setlist[description]{topsep = 0.2em, listparindent = \parindent, font = \normalfont,
  itemsep = 0em}

\newcommand{\mi}{\mathrm{i}}
\newcommand{\me}{\mathrm{e}}
\newcommand{\dd}{\mathop{}\!\mathrm{d}}

\newcommand{\bbC}{\mathbb{C}}
\newcommand{\bbD}{\mathbb{D}}
\newcommand{\bbR}{\mathbb{R}}

\newcommand{\norm}[1]{\|#1\|}

\renewcommand{\Re}{\operatorname{Re}}
\renewcommand{\Im}{\operatorname{Im}}

\DeclareMathOperator{\res}{res}

\allowdisplaybreaks

\begin{document}

\setcounter{chapter}{6}
\chapter{The Zeta Function and Prime Number Theorem}

\section{Exercises}

\begin{enumerate}
  \item Suppose that $\{a_n\}_{n=1}^\infty$ is a sequence of real numbers such that the partial sums
  \[ A_n = a_1 + \cdots + a_n \]
  are bounded. Prove that the Dirichlet series
  \[ \sum_{n=1}^\infty \frac{a_n}{n^s} \]
  converges for $\Re(s) > 0$ and defines a holomorphic function in this half-plane.
  \begin{proof}
    We first apply the Abel transform,
    \begin{align*}
      \sum_{n=1}^N \frac{a_n}{n^s} &= \sum_{n=1}^N \frac{A_n - A_{n-1}}{n^s}\\
      &= \frac{A_N}{(N+1)^s} + \sum_{n=1}^N A_n \left(\frac 1{n^s} - \frac 1{(n+1)^s}\right),
    \end{align*}
    note that $A_n$ is bounded, say, $|A_n|\leq C$, we have $A_N/(N+1)^s \to 0$,
    and
    \begin{align*}
      &\quad \sum_{n=1}^\infty \left| A_n \left( \frac 1{n^s} - \frac 1{(n+1)^s}\right) \right|\\
      &\leq C\sum_{n=1}^\infty \frac 1{n^{\sigma + 1}} < \infty,
    \end{align*}
    the summation converges. This also gives an uniform convergence for each $\Re(s) \geq \sigma$,
    thus the sum is analytic.
  \end{proof}
  \item The following links the multiplication of Dirichlet series with the divisibility
  properties of their coefficients.
  \begin{enumerate}
    \item Show that if $\{a_m\}$ and $\{b_k\}$ are two bounded sequences of complex numbers, then
    \begin{center}
      $\displaystyle \left(\sum_{m=1}^\infty \frac{a_m}{m^s}\right)
      \left(\sum_{k=1}^\infty \frac{b_k}{k^s}\right) = \sum_{n=1}^\infty \frac{c_n}{n^s}$
      \quad where $c_n = \sum_{mk=n}a_m b_k$.
    \end{center}
    The above series converge absolutely when $\Re(s) > 1$.
    \begin{proof}
      When the two sequences are bounded, say, $|a_m|, |b_k|\leq C$, we have
      \[ \sum_{m=1}^\infty \sum_{k=1}^\infty \left| \frac{a_m}{m^s}
      \frac{b_k}{k^s} \right|
      \leq \left(\sum_{m=1}^\infty \frac{C}{n^{\sigma}}\right)^2 < \infty, \]
      so the summation absolutely converges for arbitrary order, when $\sigma>1$ we have
      \begin{align*}
        &\quad \sum_{m=1}^\infty \sum_{k=1}^\infty \frac{a_m}{m^s}
        \frac{b_k}{k^s}\\
        &= \sum_{n=1}^\infty \sum_{mk=n} \frac{a_m}{m^s}
        \frac{b_k}{k^s}\\
        &= \sum_{n=1}^\infty \frac{c_n}{n^s}. \qedhere
      \end{align*}
    \end{proof}
    \item Prove as a consequence that one has
    \begin{center}
      $\displaystyle (\zeta(s))^2 = \sum_{n=1}^\infty \frac{d(n)}{n^s}$
      \quad and \quad $\displaystyle \zeta(s)\zeta(s-a) = \sum_{n=1}^\infty \frac{\sigma_a(n)}{n^s}$
    \end{center}
    for $\Re(s) > 1$ and $\Re(s - a) > 1$, respectively. Here $d(n)$ equals the number
    of divisors of $n$, and $\sigma_a(n)$ is the sum of the $a^{\text{th}}$ powers of divisors of $n$. In particular,
    one has $\sigma_0(n) = d(n)$.
    \begin{proof}
      When $\Re(s) > 1$ and $\Re(s-a)>1$, we have the absolute convergence of
      \[ \zeta(s)\zeta(s-a) = \sum_{n=1}^\infty \sum_{km=n} \frac{1}{k^sm^{s-a}}
      = \sum_{n=1}^\infty \frac{\sigma_a(n)}{n^s}. \qedhere \]
    \end{proof}
  \end{enumerate}
  \setcounter{enumi}{4}
  \item Consider the following function
  \[ \tilde \zeta(s) = 1 - \frac 1{2^s} + \frac 1{3^s} - \cdots = \sum_{n=1}^\infty \frac{(-1)^{n+1}}{n^s}. \]
  \begin{enumerate}
    \item Prove that the series defining $\tilde \zeta(s)$ converges for $\Re(s) > 0$ and defines a
    holomorphic function in that half-plane.
    \begin{proof}
      Since the partial sum of $(-1)^{n+1}$ is bounded, exercise 1 applies.
    \end{proof}
    \item Show that for $s > 1$ one has $\tilde \zeta(s) = (1 - 2^{1-s})\zeta(s)$.
    \begin{proof}
      For $s>1$, exercise 2 applies.
    \end{proof}
    \item Conclude, since $\tilde \zeta$ is given as an alternating series, that $\zeta$ has no zeros
    on the segment $0 < \sigma < 1$. Extend this last assertion to $\sigma = 0$ by using the functional equation.
    \begin{proof}
      Since
      \[ \tilde \zeta(s) = \sum_{n=1}^\infty \left(\frac 1{(2n-1)^s} - \frac 1{(2n)^s}\right), \]
      we have $\tilde \zeta(s) > 0$ for $0 < s < 1$, then $\zeta(s)\neq 0$.
      Since $\zeta(s) = \Gamma((1-s)/2)\pi^{s-1/2}\zeta(1-s)/\Gamma(s/2)$, we know $\zeta$ is nonzero at $s=0$.
    \end{proof}
  \end{enumerate}
  \setcounter{enumi}{6}
  \item Show that the function
  \[ \xi(s) = \pi^{-s/2} \Gamma(s/2) \zeta(s) \]
  is real when $s$ is real, or when $\Re(s) = 1/2$.
  \begin{proof}
    Since $\tilde \zeta(s)$ is real for $s > 0$ and $s\neq 1$, and $\Gamma(s)$ is real for $s>0$,
    we have $\xi(s)$ is real for $s > 0, s\neq 1$. Since $\xi(s) = \xi(1-s)$, we have
    $\xi(s)$ is real on the real line.

    Then consider the series expansion at $\xi(s) = \sum_n a_n (s-1/2)^n$ with
    radius of convergence $1/2$, we must have $a_n\in\bbR$.
    Since $\xi(s) = \xi(1-s)$, we have $a_n = (-1)^n a_n$, so $a_n = 0$ for $n$ odd, then we must
    have $\xi(s) \in \bbR$ for $s = 1/2 + \mi t$ where $|t| < 1/2$. We can repeat this argument
    on the expansion of a larger $t$, then we know that $\xi(s) \in \bbR$ for $\Re s = 1/2$.
  \end{proof}
  \item The function $\zeta$ has infinitely many zeros in the critical strip. This can be seen as follows.
  \begin{enumerate}
    \item Let
    \begin{center}
      $F(s) = \xi(1/2 + s)$, \quad where \quad $\xi(s) = \pi^{-s/2}\Gamma(s/2)\zeta(s)$.
    \end{center}
    Show that $F(s)$ is an even function of $s$, and as a result, there exists $G$ so
    that $G(s^2) = F(s)$.
    \begin{proof}
      We already proved this locally in the last problem. Note that $\xi(s)=\xi(1-s)$ gives $F(s) = F(-s)$, we can take
      $G(s) = F(\sqrt s)$ whenever branch we take.
    \end{proof}
    \item Show that the function $(s - 1)\zeta(s)$ is an entire function of growth order $1$,
    that is
    \[ |(s-1) \zeta(s)| \leq A_\epsilon \me^{a_\epsilon |s|^{1+\epsilon}}. \]
    As a consequence $G(s)$ is of growth order $1/2$.
    \begin{proof}
      We know that $\zeta(s)$ has the only simple pole at $s=1$, so $(s-1)\zeta(s)$ is entire.
      When $\Re s \geq 1/2$, we know that $\zeta(s) = O(|t|^{1/2 + \epsilon})$ for $|t|\geq 1$.
      Since
      \[ \zeta(s) = \frac{\Gamma\left(\frac{1-s}{2}\right)}{\zeta\left(\frac s 2\right)}
      \pi^{s-1/2} \zeta(1-s), \]
      we know $(s-1)\zeta(s)$ has growth order $\leq 1$. Thus $F(s)$ has growth order $\leq 1$.
      Along the real line we know that $F(s)$ has growth order $1$. Thus $G(s)$ has growth order $1/2$.
    \end{proof}
    \item Deduce from the above that $\zeta$ has infinitely many zeros in the critical strip.
    \begin{proof}
      If $\zeta$ has only finite zeros on the critical strip, $F$ is entire with finite zeroes.
      Hadamard's theorem tells that we have the form
      \[ F(s) = \me^{g(z)} \prod_{k=1}^n (x-a_k), \]
      where $g$ is a polynomial with $\deg \leq 0$. Then $F$ is a polynomial, contradiction.
    \end{proof}
  \end{enumerate}
  \setcounter{enumi}{10}
  \item Let
  \[ \varphi(x) = \sum_{p\leq x} \log p \]
  where the sum is taken over all primes $\leq x$. Prove that the following are equivalent
  as $x \to \infty$:
  \begin{enumerate}[label={(\roman*)}]
    \item $\varphi(x)\sim x$,
    \item $\pi(x) \sim x/\log x$,
    \item $\psi(x)\sim x$,
    \item $\psi_1(x)\sim x^2/2$.
  \end{enumerate}
  \begin{proof}
    We already proved (iv) $\implies$ (iii) $\implies$ (ii).

    (ii) $\implies$ (i): If $\pi(x) \sim x/\log x$, we have
    \begin{align*}
      \limsup_{x\to \infty} \frac{\varphi(x)}{x} & \leq \limsup_{x\to \infty} \frac{\pi(x)\log x}{x} = 1,
    \end{align*}
    and since
    \[ \varphi(x) \geq \sum_{x^\alpha < p \leq x} \log p \geq (\pi(x)-\pi(x^\alpha))\cdot \alpha\log x, \]
    we have
    \[ \liminf_{x\to \infty} \frac{\varphi(x)}{x} \geq \alpha \]
    for all $\alpha < 1$, in conclusion, $\varphi(x)\sim x$.

    (i) $\implies$ (iii): Only need to notice that
    \[ \psi(x) - \varphi(x) = \sum_{p^m \leq x, m\geq 2}\log p \leq \sqrt x \log x, \]
    we have $\psi(x)\sim x$.

    (iii) $\implies$ (iv): Stolz--Ces\`aro theorem tells
    \[ \lim_{x\to \infty} \frac{\psi_1(x)}{x^2/2} = 
    \lim_{x\to \infty} \frac{\psi(x)}{x+1/2} = 1. \qedhere \]
  \end{proof}
\end{enumerate}

\end{document}
